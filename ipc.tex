\documentclass[11pt]{article}
\usepackage[utf8]{inputenc}
\usepackage[spanish]{babel}
\usepackage[T1]{fontenc}
\usepackage{lmodern}
\usepackage{microtype}
\usepackage{parskip}
\usepackage[a4paper, margin=2.5cm]{geometry}
\usepackage{titlesec}
\usepackage{setspace}
\usepackage{enumitem}
\usepackage{xcolor}
\usepackage{tcolorbox}
\usepackage{hyperref}
\hypersetup{
    colorlinks=true,
    linkcolor=blue!50!black,
    urlcolor=blue!60!black,
    citecolor=blue!50!black
}

% Formato de título elegante
\titleformat{\section}
  {\normalfont\Large\bfseries\color{blue!50!black}}
  {\thesection}{1em}{}

\titleformat{\subsection}
  {\normalfont\large\bfseries\color{blue!80!black}}
  {\thesubsection}{1em}{}

\renewcommand{\maketitle}{
    \begin{center}
        \vspace*{1cm}
        {\Huge \bfseries\color{blue!70!black} IPC y Sockets \par}
        \vspace{0.5cm}
        {\large \textbf{Niurka Vanesa Yupanqui} \par}
        \vspace{0.3cm}
        {\small 04 de julio de 2025 \par}
        \vspace{0.5cm}
        \hrule height 1pt
        \vspace{1cm}
    \end{center}
}

\begin{document}
\onehalfspacing
\pagestyle{empty}
\maketitle

\section*{Introducción}
\noindent
Como parte de la tarea asincrónica sobre \textbf{IPC y sockets}, realicé la ejecución de distintos programas incluidos en las carpetas proporcionadas. Esta actividad me permitió observar de manera práctica cómo funciona la comunicación entre procesos, la creación de procesos hijos y la sincronización entre hilos.

\section*{Exploración de carpetas y programas}

\begin{tcolorbox}[colback=blue!5!white, colframe=blue!40!black, title=Carpeta \texttt{procesos}]
Ejecuté el archivo \texttt{fork.c}, el cual utiliza la función \texttt{fork()} para crear procesos hijos. Al compilarlo y ejecutarlo, noté cómo el mismo código se ejecuta en paralelo tanto en el proceso padre como en el hijo. 

También trabajé con \texttt{sumSec.c} y \texttt{sumPar.c}, que permiten comparar la ejecución secuencial con una paralela utilizando múltiples procesos.
\end{tcolorbox}

\begin{tcolorbox}[colback=blue!5!white, colframe=blue!40!black, title=Carpeta \texttt{cinco/}]
Ejecuté el archivo \texttt{cinco.c}, una implementación del problema de los cinco filósofos. Este ejemplo me permitió analizar cómo se gestiona la sincronización entre procesos que comparten recursos, y cómo se evita el bloqueo o el acceso simultáneo no controlado.
\end{tcolorbox}

\begin{tcolorbox}[colback=blue!5!white, colframe=blue!40!black, title=Carpeta \texttt{threads/}]
Trabajé con el archivo \texttt{tres\_threads.c}, donde observé cómo se crean varios hilos que se ejecutan al mismo tiempo dentro de un mismo proceso. Para su compilación utilicé la librería \texttt{-lpthread}. También observé cómo los hilos imprimen mensajes de forma concurrente, mostrando un ejemplo claro de paralelismo.
\end{tcolorbox}

\section*{Reflexión final}
Esta práctica me ayudó a comprender mejor los conceptos de procesos, hilos y concurrencia. Me familiaricé con herramientas como \texttt{gcc}, \texttt{make} y el uso de bibliotecas para programación con hilos. A través de los ejemplos ejecutados, pude ver la importancia de la sincronización y la forma en que los procesos se comunican y trabajan en conjunto en un entorno de sistemas operativos.

\vspace{1cm}
\begin{flushright}
    \textbf{Niurka Vanesa Yupanqui}
\end{flushright}

\end{document}